\documentclass[12pt]{article}
\usepackage[utf8]{inputenc}
\usepackage[english]{babel}
\usepackage{bbding}
\decimalpoint
\usepackage[spanish]{babel}
\usepackage{amsmath}
\usepackage{amsthm}
\usepackage{amssymb}
\usepackage{graphicx}
\usepackage[margin=0.9in]{geometry}
\usepackage{fancyhdr}
\usepackage[inline]{enumitem}
\usepackage{float}
\usepackage{cancel}
\usepackage{minted}
\usepackage{bigints}
\usepackage{color}
\usepackage{xcolor}
\usepackage{subfig}
\usepackage{listingsutf8}
\usepackage{algorithm}
\usepackage{tocloft}
\usepackage[none]{hyphenat}
\usepackage{graphicx}
\usepackage{grffile}
\usepackage{tabularx}
\usepackage{hyperref}
\usepackage[nottoc,notlot,notlof]{tocbibind}
\usepackage{times}
\usepackage{color}
\definecolor{gray97}{gray}{.97}
\definecolor{gray75}{gray}{.75}
\definecolor{gray45}{gray}{.45}
\renewcommand{\cftsecleader}{\cftdotfill{\cftdotsep}}
\pagestyle{fancy}
\setlength{\headheight}{15pt} 
\lhead{Cryptoanalysis}
\rhead{\thepage}
\lfoot{ESCOM-IPN}
\renewcommand{\footrulewidth}{0.5pt}
\setlength{\parskip}{0.5em}
\newcommand{\ve}[1]{\overrightarrow{#1}}
\newcommand{\abs}[1]{\left\lvert #1 \right\lvert}
\date{February 6, 2019}
\title{Instalación de Netbeans}
\author{Reporte 1}
\usepackage{minted}
\setminted{
    style=emacs,
    breaklines=true
}

%Permite crear columnas en el documento
\usepackage{multicol} 
\usepackage{color}
\usepackage{comment}
\newcommand{\tabitem}{~~\llap{\textbullet}~~}
\newcommand{\subtabitem}{~~~~\llap{\textbullet}~~}

\usepackage{cmbright}                               % Font
\bibliographystyle{IEEEtran}
\begin{document}
    \begin{titlepage}
		\begin{center}
			% Upper part of the page. The '~' is needed because \\
			% only works if a paragraph has started.
			
			\noindent
			\begin{minipage}{0.5\textwidth}
				\begin{flushleft} \large
					\includegraphics[width=0.3\textwidth]{../ipn.png}
				\end{flushleft}
			\end{minipage}%
			\begin{minipage}{0.55\textwidth}
				\begin{flushright} \large
					\includegraphics[width=0.7\textwidth]{../escom.png}
				\end{flushright}
			\end{minipage}
			
			\textsc{\LARGE Instituto Politécnico Nacional}\\[0.5cm]
			
			\textsc{\Large Escuela Superior de Cómputo}\\[1cm]
			
			% Title
			
			{ \huge Session 3 \\[1cm] }
			
			{ \Large Club de Algoritmia} \\[1cm]
			
			{ \Large SS  } \\[1cm]
			
			\noindent
			\begin{minipage}{0.5\textwidth}
				\begin{flushleft} \large
					\emph{Students:}\\
					
					\begin{tabular}{ll}
				     Nicolás Sayago Abigail\\

				\end{tabular}
				\end{flushleft}
			\end{minipage}%
			\begin{minipage}{0.5\textwidth}
				\begin{flushright} \large
					\emph{Teacher:} \\
					Díaz Santiago Sandra  \\
				\end{flushright}
			\end{minipage}
			
			\vfill
			
			% Bottom of the page
			{\large February 27, 2019}
		\end{center}
	\end{titlepage}
	
    \tableofcontents
    \newpage
    
    % ----------------------------------------------------
    %               PROBLEMAS OMEGA UP     
    % ----------------------------------------------------
    \section{Problemas OmegaUp}
         
         % ================================
         %       MODA, MEDIANA, MEDIA
         % ================================
          \subsection{Moda, Mediana, Media}
            $\rightarrow$\textbf{URL: }\url{http://www.latex-project.org/}
            
            \subsubsection{Solución:}
                \begin{itemize}
                    \item Revisar Media, en el peor de los casos, la suma puede llegar a ser $10^{10}$. Para un entero solo puedo hacer $2^{31}-1$, empieza a salir números negativos, entonces lleva a una mala respuesta. Un \textbf{long long int} puede llegar a tener: $2^{63}-1 = 10^{18}$.
                    
                    \item Optimización de entrada:
                        \inputminted{cpp}{Code/OmegaUp/optimizacionIn.cpp}
                \end{itemize}
            
            \subsubsection{Código:}
                \inputminted{cpp}{Code/MMM.cpp}
                
         % ================================
         %          SUMA
         % ================================
          \subsection{Suma Triguereana}
            $\rightarrow$\textbf{URL: }\url{http://www.latex-project.org/}
            
            \subsubsection{Solución:}
                Aplicar algoritmo de suma "a manita". Tomando en cuenta acarreos. (\textbf{Stoi} significa: Stringo to int, y \textbf{Stol} significa: String to long long int).
            
            \subsubsection{Código:}
                \inputminted{cpp}{Code/suma.cpp}
        
        
        
        
    
    
\end{document}